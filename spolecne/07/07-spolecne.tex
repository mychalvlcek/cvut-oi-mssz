%!TEX root=../oi-magistr-spolecne.tex
\section[KO - ILP, toky]{Metoda větví a mezí. Algoritmy pro celočíselné lineární programování. Formulace optimalizačních a rozhodovacích problémů pomocí celočíselného lineárního programování. Toky a řezy. Multi-komoditní toky.}

\subsection*{ILP - celočíselné lineární programování}

Úloha celočíselného lineárního programování (LP) je zadána maticí $\A \in \R^{m \times n}$ a vektory $b \in \R^m, c \in \R^n$. Cílem je najít takový vektor $x \in \Z^n$, že platí $\A \cdot x \leq b$ a $c^T \cdot x$ je maximální.

Obvykle se celočíselné lineární programování zapisuje ve tvaru:

	$$max(c^T \cdot x : \A \cdot x \leq b, x \in Z^n)$$
	
Pokud bychom takovou úlohu řešili pomocí lineárního programování s tím, že bychom výsledek zaokrouhlili, nejenom že bychom neměli zaručeno že výsledné řešení bude optimální ale ani to, zda bude přípustné. 
Zatímco úloha LP je řešitelná v polynomiálním čase, úloha ILP je tzv. \hyperref[heading:05-npc]{NP-těžká} (NP-hard), neboli není znám algoritmus, který by vyřešil libovolnou instanci této úlohy v polynomiálním čase. Protože prostor řešení ILP není konvexní množina, nelze přímo aplikovat metody konvexní optimalizace.

\subsubsection*{Formulace optimalizačních a rozhodovacích problémů pomocí ILP.}

\subsection*{Algoritmy pro celočíselné lineární programování}
\begin{enumerate}
	\item Výčtové metody (Enumerative Methods)
	\item Metoda větví a mezí (Branch and Bound)
	\item Metody sečných nadrovin (Cutting Planes Methods)
\end{enumerate}

\subsubsection*{Výčtové metody (Enumerative Methods)}
Výpočet je založen na prohledávání oblasti zahrnující všechna přípustná řešení [1, 2]. Vzhledem k celočíselnému omezení proměnných je počet těchto řešení konečný ale jejich počet je extrémně vysoký. Proto je tato metoda vhodná pouze pro malé problémy s omezeným počtem diskrétních proměnných. Postup je možno zobecnit na úlohu MIP (mixed IP) tak, že ke každé kombinaci diskrétních proměnných je vyřešena úloha LP kde jsou diskrétní proměnné považovány za konstanty. \cite{ko:ilp-sucha}

\subsubsection*{Metoda větví a mezí (Branch and Bound)}
text + obr s prikladem

\subsubsection*{Metody sečných nadrovin (Cutting Planes Methods)}
Další skupinou algoritmů jsou metody sečných nadrovin (cutting plane method), založené podobně jako metoda větví a mezí na opakovaném řešení úlohy LP. Výpočet je prováděn iterativně tak, že v každém kroku je přidána další omezující podmínka zužující oblast přípustných řešení. Každá nová omezující podmínka musí splňovat tyto vlastnosti:

\begin{enumerate}
	\item Optimální řešení nalezené pomocí LP se stane nepřípustným.
	\item Žádné celočíselné řešení přípustné v předchozím kroku se nesmí stát nepřípustným. 
\end{enumerate}

Nové omezení splňující tyto vlastnosti je přidáno v každé iteraci. Vzniklý ILP program je vždy znovu řešen jako úloha LP. Proces je opakován, dokud není nalezeno přípustné celočíselné řešení. Konvergence takovéhoto algoritmu potom závisí na způsobu přidávání omezujících podmínek. Mezi nejznámější metody patří Dantzigovi řezy (Dantzig cuts) a Gomoryho řezy (Gomory cuts). \cite{ko:ilp-sucha}

\subsection*{Formulace optimalizačních a rozhodovacích problémů pomocí celočíselného lineárního programování}

\subsection*{Toky a řezy}

\subsection*{Multi-komoditní toky}