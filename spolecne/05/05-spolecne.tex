%!TEX root=../oi-magistr-spolecne.tex
\section[TAL - Algoritmus, P, NP]{Algoritmus, správnost algoritmu, složitost algoritmu, složitost úlohy, třída P, třída NP.}

\paragraph{Algoritmus.}
\label{algoritmus}
\emph{Algoritmem} rozumíme dobře definovaný proces, tj. posloupnost výpočetních kroků, který přijímá hodnoty (zadání, vstup) a vytváří hodnoty (řešení, výstup).

\paragraph{Správnost algoritmu}
K ověření správnosti algoritmu je třeba ověřit 2 věci:
\begin{enumerate}[itemsep=0pt]
    \item algoritmus se na každém vstupu zastaví
    \item algoritmus po zastavení vydá správný výstup - řešení
\end{enumerate}

\paragraph{Variant.}
Pro důkaz faktu, že se algoritmus na každém vstupu zastaví, je založen na nalezení tzv. \emph{variantu}. Variant je hodnota udaná přirozeným číslem, která se během práce algoritmu snižuje až nabude nejmenší možnou hodnotu (a tím zaručuje ukončení algoritmu po konečně mnoha krocích).

Příklad:

\paragraph{Invariant.}
\emph{Invariant}, též \emph{podmíněná správnost algoritmu}, je tvrzení, které:
\begin{itemize}[itemsep=0pt]
    \item platí před vykonáním prvního cyklu algoritmu, nebo po prvním vykonání cyklu
    \item platí-li před vykonáním cyklu, platí i po jeho vykonání
    \item při ukončení práce algoritmu zaručuje správnost řešení
\end{itemize}

Příklad:

\subsection*{Složitost algoritmu}

\subsection*{Složitost úlohy - třídy složitosti}
Úlohy/algoritmy chceme nějak zařadit, a tak vznikají třídy složitosti, redukce, atd.

\subsection*{Třída složitosti - P}

\subsection*{Třída složitosti - NP}

Otázka obsahuje texty, úryvky a definice z \cite{tal:demlova}.
