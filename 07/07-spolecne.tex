%!TEX root=../oi-magistr-spolecne.tex
\section[TAL - Turingovy stroje]{Turingovy stroje, rekurzivní a rekurzivně spočetné jazyky, algoritmicky neřešitelné úlohy.}

\paragraph{Turingovy stroje} viz předchozí otázky. TODO: možná nějaký příklad TM

\subsection{Rekurzivní jazyky}
Řekneme, že jazyk $L$ je \textit{rekurzivní}, jestliže existuje Turingův stroj $M$, který rozhoduje jazyk $L$.

Připomeńme, že Turingův stroj $M$ rozhoduje jazyk $L$ znamená, že jej přijímá a na každém vstupu se zastaví (buď úspěšně nebo neúspěšně).

\subsection{Rekurzivně spočetné jazyky}
Řekneme, že jazyk $L$ je \textit{rekurzivně spočetný}, jestliže existuje Turingův stroj $M$, který tento jazyk přijímá.

Jinými slovy, $M$ se pro každé slovo $w$, které patří do $L$, úspěšně zastaví a pro slovo $w$, které nepatří do $L$ se buď zastaví neúspěšně nebo se nezastaví vůbec.

Jazykům, které \textbf{nejsou rekurzivní}, také říkáme, že jsou \textit{algoritmicky neřešitelné} nebo \textit{nerozhodnutelné}. Obdobně mluvíme o úlohách, které jsou nerozhodnutelné nebo \textbf{algoritmicky neřešitelné}.

\begin{itemize}
\item Diagonální jazyk
\item Univerzální jazyk
\end{itemize}
